\documentclass[11pt,a4paper]{article}

% ── Packages ────────────────────────────────────────────────
\usepackage[a4paper, margin=2.5cm]{geometry}
\usepackage{setspace}
\doublespacing
\usepackage{graphicx}
\usepackage{amsmath}
\usepackage{amssymb}
\usepackage{siunitx}
\usepackage{booktabs}
\usepackage{hyperref}
\usepackage{cite}
\usepackage{caption}
\usepackage{subcaption}
\usepackage{float}
\usepackage{parskip}
\usepackage{fancyhdr}
\usepackage{titlesec}
\usepackage{xcolor}
\usepackage[T1]{fontenc}

% ── Page style ───────────────────────────────────────────────
\pagestyle{fancy}
\fancyhf{}
\rhead{\small IRNAS KORUZA FSO Communicator for QKD}
\cfoot{\thepage}
\renewcommand{\headrulewidth}{0.4pt}

\hypersetup{
    colorlinks=true,
    linkcolor=black,
    citecolor=black,
    urlcolor=blue
}

% ── Title info ───────────────────────────────────────────────
\title{
    \vspace{-1cm}
    {\large Year 4 Research Project}\\[0.5em]
    {\Large \textbf{Experimental Characterisation of a Terrestrial Free-Space Optical
    Link for Quantum Key Distribution}}\\[1em]
    {\normalsize Heriot-Watt University}\\
    {\normalsize Submission date: }
}
\author{
    Todd Blacklaw, Student ID: H00368671\\
    Supervisor: Alessandro Fedrizzi 
}
\date{}

% ── Helpers ─────────────────────────────────────────────────
\newcommand{\cn}{$C_n^2$}
\newcommand{\kext}{$k_\mathrm{ext}$}
\newcommand{\etaatm}{$\eta_\mathrm{atm}$}
\newcommand{\todo}[1]{\textcolor{red}{\textbf{[TODO: #1]}}}

% ============================================================
\begin{document}

\maketitle
\thispagestyle{empty}
\newpage

% ── Abstract ─────────────────────────────────────────────────
\begin{abstract}
Free-space optical (FSO) communication offers fibre-like data rates without
physical infrastructure, making it a compelling platform for both urban
backhaul and, increasingly, quantum key distribution (QKD). This project
constructs and characterises a low-cost terrestrial FSO terminal based on
the open-source IRNAS KORUZA design, replacing the original custom
printed circuit board with off-the-shelf components centred on a Raspberry
Pi controller. The optical link operates at \SI{1550}{nm} via a standard
small-form-factor pluggable transceiver and is directed using a three-axis
unipolar stepper-motor pointing, acquisition, and tracking system. The
atmospheric channel is modelled using the libRadtran radiative-transfer
suite, which yields extinction coefficients for three Scottish weather
scenarios (clear summer, $k_\mathrm{ext} = \SI{0.036}{km^{-1}}$; clear
winter, $k_\mathrm{ext} = \SI{0.040}{km^{-1}}$; mist,
$k_\mathrm{ext} = \SI{0.237}{km^{-1}}$) via Monte Carlo photon transport.
These transmittance values are propagated through three complementary
Qiskit quantum-circuit simulations: a BB84 channel-feasibility study, an
intercept-resend eavesdropper-detection model, and a log-normal scintillation
fading analysis. At a \SI{500}{m} deployment distance, all weather scenarios
yield quantum bit error rates well below the \SI{11}{\percent}
Shor--Preskill security threshold (1.3--1.4\%), with secure key rates of
5.3--5.9~kbps. An intercept-resend attack raises the QBER to $\sim$\SI{40}{\percent},
demonstrating unambiguous eavesdropper detectability. Scintillation fading
analysis shows that even under strong turbulence (scintillation index $= 1.5$),
the fraction of time the link enters the insecure region is below
\SI{0.4}{\percent}, confirming that a KORUZA-based FSO terminal is a
viable platform for short-range metropolitan QKD with straightforward
hardware additions.
\end{abstract}

\newpage
\tableofcontents
\newpage

% ============================================================
\section{Introduction}
\label{sec:intro}

The demand for high-bandwidth, secure, and rapidly deployable communication
links has grown substantially in recent years. Conventional radio-frequency
(RF) systems are constrained by spectrum congestion and licensing costs,
while fibre optic deployment is prohibitively expensive or impractical in
last-mile urban scenarios and areas of difficult terrain
\cite{khalighi2014survey, barrios2012wireless}. Free-space optical (FSO)
communication has emerged as a compelling alternative: it offers data rates
comparable to fibre (Gbps to Tbps), operates on an unlicensed spectrum,
consumes relatively low power, and provides an inherent degree of physical
security through its narrow, line-of-sight beam geometry
\cite{chan2006free}.

Despite these advantages, the reliability of terrestrial FSO links is
limited primarily by the atmosphere. Random fluctuations in the refractive
index of air—caused by solar heating, wind shear, and resulting thermal
inhomogeneities—distort and attenuate the propagating optical wavefront in
a process broadly termed atmospheric turbulence \cite{fante1975electromagnetic}.
The principal manifestations of turbulence are scintillation (rapid
irradiance fluctuations at the receiver), beam wander (random displacement
of the beam centroid), and wavefront distortion, all of which degrade link
performance and increase the bit error rate (BER) \cite{khalighi2014survey}.

A second, growing motivation for robust FSO links is their role as the
physical layer for quantum key distribution (QKD). QKD protocols such as
BB84 offer provably secure key exchange based on the laws of quantum
mechanics \cite{bennett1984}. While fibre-based QKD is mature, its
practicality is curtailed by exponential photon loss with distance.
Satellite-based QKD, demonstrated dramatically by the Micius satellite
at distances up to 1200~km using decoy-state BB84 \cite{liao2017,lo2005},
and metropolitan FSO QKD \cite{krzic2023metropolitan,bedington2017}
present natural solutions, but the same
atmospheric effects that degrade classical links represent a critical barrier
for quantum communication. Signal loss from beam wander reduces the rate of
single-photon delivery, while wavefront distortion can increase the quantum
bit error rate (QBER) by mixing polarisation or phase states, reducing the
margin within which an eavesdropper can be detected
\cite{erven2008entangled}.

This project addresses both challenges simultaneously. A low-cost,
open-source FSO terminal is constructed and characterised, based on the
IRNAS KORUZA platform \cite{koruza}, with the dual aim of (1) quantifying
the impact of Scottish atmospheric conditions on FSO link performance, and
(2) determining the feasibility and limitations of the resulting channel
for BB84 QKD. The atmospheric channel is modelled using the libRadtran
Monte Carlo radiative-transfer code \cite{emde2016libradtran}, and the
quantum channel behaviour is simulated using Qiskit quantum circuits
incorporating physically motivated noise models.

The remainder of this report is structured as follows.
Section~\ref{sec:background} reviews the relevant background theory spanning
atmospheric turbulence physics, FSO channel modelling, and QKD principles.
Section~\ref{sec:methods} describes the hardware construction and the
computational methods applied. Section~\ref{sec:results} presents the
atmospheric and quantum simulation results. Section~\ref{sec:discussion}
interprets these findings in the context of the literature and the project
objectives. Section~\ref{sec:conclusions} states the conclusions and
directions for further work.

% ============================================================
\section{Background Theory}
\label{sec:background}

\subsection{Atmospheric Turbulence and FSO Propagation}

The theoretical framework for electromagnetic wave propagation through
random media was established by Fante \cite{fante1975electromagnetic} and
extended by Tatarskii \cite{tatarskii1961}. Atmospheric turbulence
originates in thermal inhomogeneities driven principally by solar heating
of the Earth's surface and subsequent wind mixing, which produce random
fluctuations in air temperature and pressure. These fluctuations modulate
the atmospheric refractive index $n(\mathbf{r}, t)$, which can be
decomposed as:
\begin{equation}
    n(\mathbf{r}, t) = \langle n \rangle + n_1(\mathbf{r}, t),
    \label{eq:refrac}
\end{equation}
where $\langle n \rangle \approx 1$ is the mean and $n_1$ is a zero-mean
random perturbation. Following the Kolmogorov cascade model, turbulent
energy is injected at a large outer scale $L_0$ (metres to hundreds of
metres) and dissipates as heat at an inner scale $l_0$ (typically
millimetres). Within this inertial subrange, the structure function of the
refractive index fluctuations obeys:
\begin{equation}
    D_n(r) = \langle [n(\mathbf{r}_1) - n(\mathbf{r}_2)]^2 \rangle
            = C_n^2\, r^{2/3},
    \label{eq:structfunc}
\end{equation}
where $r = |\mathbf{r}_1 - \mathbf{r}_2|$ and $C_n^2$ is the
\emph{refractive-index structure parameter} (units: m$^{-2/3}$), the
primary scalar measure of turbulence strength \cite{khalighi2014survey}.
Typical values for a horizontal terrestrial path range from
$C_n^2 \sim 10^{-16}$~m$^{-2/3}$ (weak turbulence) to
$C_n^2 \sim 10^{-13}$~m$^{-2/3}$ (strong turbulence) \cite{motlagh2008}.

As an optical beam propagates through this medium, the refractive
inhomogeneities cause three principal effects. \emph{Scintillation}
(intensity fading) arises from small-scale eddies comparable to or smaller
than the beam diameter, which scatter and diffract light, producing a
random irradiance pattern at the receiver. \emph{Beam wander} is caused by
large-scale eddies that act as refractive wedges, displacing the entire
beam centroid from its intended path \cite{anthonisamy2016performance}. \emph{Wavefront distortion} describes
higher-order phase aberrations that degrade coherence. All three phenomena
increase the BER and, for quantum links, the QBER
\cite{barrios2012wireless}.

\subsection{Statistical Channel Models}

For the log-normal turbulence regime (weak to moderate, propagation
distances $\lesssim 1$~km at near-infrared wavelengths), the instantaneous
received irradiance $I$ is modelled as a log-normal random variable:
\begin{equation}
    I = I_0 \exp(2\chi),
    \label{eq:lognormal}
\end{equation}
where $\chi$ is a Gaussian random variable with mean $\mu_\chi = -\sigma_\chi^2$
(so that $\langle I \rangle = I_0$) and variance $\sigma_\chi^2$. The
\emph{scintillation index} (SI), which quantifies the normalised variance
of irradiance, is related to the log-normal parameters by:
\begin{equation}
    \mathrm{SI} = \frac{\langle I^2 \rangle - \langle I \rangle^2}{\langle I \rangle^2}
                = \exp(4\sigma_\chi^2) - 1.
    \label{eq:SI}
\end{equation}
In the present work, SI values of 0.1, 0.5, and 1.5 are used to represent
weak, moderate, and strong turbulence conditions, respectively
\cite{andrews2005laser}.

For stronger turbulence, the Gamma-Gamma distribution, introduced by
Al-Habash, Andrews and Phillips \cite{al-habash2001} as a doubly-stochastic
model valid from weak to strong irradiance fluctuations, provides a better
fit to simulation data, and has been applied to composite channel models
incorporating dust and fog effects \cite{mohamedfso2023}. At the 500~m deployment range relevant to this
project, however, the log-normal model is well justified
\cite{khalighi2014survey}.

\subsection{Static Atmospheric Attenuation and the Beer-Lambert Law}

In addition to turbulence-induced fading, a horizontal FSO link
experiences quasi-static attenuation from molecular absorption and particle
scattering. For a homogeneous path of length $d$, the mean channel
transmittance follows the Beer-Lambert law:
\begin{equation}
    \eta_\mathrm{atm}(d) = \exp(-k_\mathrm{ext}\, d),
    \label{eq:beerlam}
\end{equation}
where $k_\mathrm{ext}$ (km$^{-1}$) is the wavelength-dependent
\emph{extinction coefficient}, which decomposes as:
\begin{equation}
    k_\mathrm{ext} = \sigma_\mathrm{Ray} + \sigma_\mathrm{Mie}
                   + \kappa_\mathrm{gas} + \kappa_\mathrm{aer},
    \label{eq:kext}
\end{equation}
representing contributions from Rayleigh scattering by air molecules,
Mie scattering by aerosols, molecular absorption, and aerosol absorption,
respectively. At \SI{1550}{nm}, Rayleigh scattering scales as
$\lambda^{-4}$ and is negligible ($<1\%$ contribution); Mie scattering by
aerosols ($r \approx \lambda$) is the dominant loss mechanism in clear air,
contributing 60--70\% of $k_\mathrm{ext}$; molecular absorption is low
because \SI{1550}{nm} falls in a well-known atmospheric transmission window
between H$_2$O and CO$_2$ absorption bands \cite{chan2006free}.

The extinction coefficient values used in this work are computed using the
libRadtran radiative-transfer package \cite{emde2016libradtran}, as
described in Section~\ref{sec:atm_model}.

\subsection{BB84 Quantum Key Distribution}

Quantum key distribution exploits quantum mechanical principles to
guarantee provably secure key exchange. The BB84 protocol, proposed by
Bennett and Brassard in 1984 \cite{bennett1984}, encodes classical bits in
the quantum states of individual photons using two conjugate bases (e.g.\
rectilinear $\{|0\rangle, |1\rangle\}$ and diagonal
$\{|{+}\rangle, |{-}\rangle\}$). Alice prepares single photons
in randomly chosen states from randomly chosen bases; Bob measures in a
randomly chosen basis. After transmission, they publicly compare bases
(but not results) and retain only the sifted key where their bases agree.

The security of BB84 rests on the no-cloning theorem: an eavesdropper
(Eve) cannot measure an unknown quantum state without disturbing it. An
intercept-resend attack, in which Eve measures every qubit and resends her
best guess, introduces a characteristic 25\% QBER on a perfect channel,
regardless of her strategy \cite{bennett1984}. The \emph{quantum bit error
rate} is thus the primary security diagnostic. The asymptotic secure key
rate under the Shor--Preskill bound is:
\begin{equation}
    R_\mathrm{secure} = R_\mathrm{sift}\,\max\!\left[0,\; 1 - 2H(e)\right],
    \label{eq:skr}
\end{equation}
where $R_\mathrm{sift} = \frac{1}{2} f_\mathrm{rep}\, \mu\,
\eta_\mathrm{atm}\, \eta_\mathrm{det}$ is the sifted key rate,
$f_\mathrm{rep}$ is the pulse repetition rate, $\mu$ is the mean photon
number per pulse, $\eta_\mathrm{det}$ is the detector efficiency, and
$H(e) = -e\log_2 e - (1-e)\log_2(1-e)$ is the binary Shannon entropy
evaluated at QBER $e$. Secret-key generation requires $e < 11\%$ (the
Shor--Preskill threshold) \cite{shor2000simple}.

\subsection{QBER Contributions in a Practical FSO-QKD Channel}

The total QBER in a practical weak-coherent-pulse (WCP) FSO-QKD system
is the sum of three contributions \cite{gobby2004quantum}:
\begin{equation}
    e = e_\mathrm{misalign} + e_\mathrm{dark} + e_\mathrm{turb},
    \label{eq:qber_total}
\end{equation}
where $e_\mathrm{misalign}$ is residual optical misalignment error
($\sim 1\%$ for a well-aligned system), $e_\mathrm{dark}$ is the
contribution from detector dark counts:
\begin{equation}
    e_\mathrm{dark} = \frac{d_\mathrm{rate}/f_\mathrm{rep}}{2\,\mu\,
    \eta_\mathrm{atm}\,\eta_\mathrm{det}},
    \label{eq:edark}
\end{equation}
and $e_\mathrm{turb}$ captures the additional error introduced when
scintillation-driven fading reduces the instantaneous transmittance below
its mean value. In the log-normal fading model, $e_\mathrm{turb}$ is
treated as a random variable whose distribution is computed via Monte Carlo
sampling of \etaatm, as described in Section~\ref{sec:qkd_sim}.

% ============================================================
\section{Materials and Methods}
\label{sec:methods}

\subsection{Hardware: KORUZA FSO Terminal}
\label{sec:hardware}

The FSO terminal constructed in this work is based on the open-source IRNAS
KORUZA design \cite{koruza}, a modular platform originally developed for
urban last-mile wireless links. The original design uses a dedicated custom
printed circuit board (PCB) to interface the Raspberry Pi controller with
the stepper motor drivers. In this implementation, the custom PCB has been
replaced entirely with commercially available off-the-shelf components,
improving repairability and reducing procurement lead times.

The mechanical housing is produced by fused-deposition modelling (FDM)
3D printing in polylactic acid (PLA), following the original KORUZA
mechanical drawings. The optical assembly is contained within this housing
and aligned along a common optical axis with the SFP port.

\subsubsection{Pointing, Acquisition and Tracking System}

Precise beam alignment is critical for FSO links: angular mispointing
introduces an additional loss factor that grows quadratically with the ratio
of angular error to beam divergence. The pointing, acquisition, and tracking
(PAT) system implemented here consists of three unipolar stepper motors
arranged to provide coarse azimuth, elevation, and roll adjustment. The
motors are driven by a ULN2003 Darlington transistor array, which interfaces
directly to the Raspberry Pi 2B general-purpose input/output (GPIO) pins
without requiring a dedicated motor controller board. This design choice
significantly reduces component count relative to the original KORUZA.

\subsubsection{Optical Data Link}

The data link is established at \SI{1550}{nm} using a TP-Link MC220L media
converter, which converts the Ethernet frame stream from the Raspberry Pi
into an optical signal via a standard small-form-factor pluggable (SFP)
transceiver module. The SFP module produces a collimated infrared beam
suitable for free-space propagation at the KORUZA designed divergence angle.
This wavelength is selected for three reasons: it falls in the principal
atmospheric transmission window (Section~\ref{sec:background}), it is
eye-safe at the power levels used, and it is compatible with
telecom-standard components, keeping system cost low.

\subsection{Atmospheric Channel Model}
\label{sec:atm_model}

The atmospheric extinction coefficient \kext~at \SI{1550}{nm} is computed
using the libRadtran software package (version 2.0.6) \cite{emde2016libradtran},
specifically the \texttt{uvspec} radiative-transfer solver operating in
Monte Carlo (MYSTIC) mode with $10^5$ photons per simulation run. The
AFGL standard atmosphere profiles are used as the baseline gas concentration
and temperature data, with aerosol properties set according to the scenario
of interest.

Three scenarios representative of Scottish atmospheric conditions are
simulated:
\begin{enumerate}
    \item \textbf{Baseline (Clear Summer):} AFGL mid-latitude summer
    atmosphere (\texttt{afglms.dat}), rural aerosol model, visibility
    \SI{23}{km}.
    \item \textbf{Scottish (Clear Winter):} AFGL mid-latitude winter
    atmosphere (\texttt{afglmw.dat}), rural aerosol model, visibility
    \SI{20}{km}.
    \item \textbf{Scottish (Mist/Drizzle):} AFGL mid-latitude winter
    atmosphere, rural aerosol model, reduced visibility \SI{4}{km} to
    simulate light mist.
\end{enumerate}

The ground-level extinction coefficient is extracted from the verbose output
of \texttt{uvspec} by parsing the \texttt{optical\_properties()} table.
The total optical depth per unit length is computed as:
\begin{equation}
    k_\mathrm{ext} = \frac{\Delta\tau_\mathrm{Ray} + \Delta\tau_\mathrm{Mie}
    + \Delta\tau_\mathrm{mol}}{\Delta z},
    \label{eq:kext_parse}
\end{equation}
where $\Delta\tau$ are the per-layer differential optical depths extracted
from the simulation output and $\Delta z = \SI{1}{km}$ is the layer
thickness. The transmittance at distance $d$ is then computed analytically
via equation~(\ref{eq:beerlam}).

\subsection{QKD Channel Simulations}
\label{sec:qkd_sim}

Three complementary quantum-channel simulations are implemented using
Qiskit 2.3 and Qiskit Aer \cite{qiskit}. In all cases, channel photon
loss is modelled as an \emph{amplitude damping channel} with damping
parameter:
\begin{equation}
    \gamma = 1 - \eta_\mathrm{atm}\,\eta_\mathrm{det},
    \label{eq:gamma}
\end{equation}
which maps the excited state $|1\rangle$ to the ground state $|0\rangle$
with probability $\gamma$, physically corresponding to photon absorption
or scattering. This is the correct quantum noise model for photon
loss \cite{nielsen2010quantum}.

\subsubsection{BB84 Feasibility Simulation}

A single-qubit circuit prepares the state $|{+}\rangle = H|0\rangle$
(representing Alice's diagonal-basis photon), applies the amplitude
damping channel, and rotates back to the $Z$ basis for measurement.
The QBER from the circuit (fraction of $|1\rangle$ outcomes) is compared
against the theoretical prediction from equations~(\ref{eq:qber_total})
and (\ref{eq:edark}). The secure key rate is then computed using
equation~(\ref{eq:skr}) for each atmospheric scenario as a function of
link distance (50~m to 5~km). System parameters are: $\mu = 0.1$,
$\eta_\mathrm{det} = 0.15$, $f_\mathrm{rep} = \SI{1}{MHz}$,
$d_\mathrm{rate} = 100$~s$^{-1}$, $e_\mathrm{misalign} = 0.01$.

\subsubsection{Eavesdropper Detection Simulation}

An intercept-resend attack is modelled by replacing the direct channel
with a two-stage circuit: Eve applies a random basis measurement
($Z$ or $X$, each with probability 0.5) and resends the post-measurement
state. When Eve's basis mismatches Alice's (50\% of intercepts), the
collapsed state is an eigenstate of the wrong basis, introducing additional
errors at Bob's receiver. Two sub-circuits are run (Eve measuring in $Z$
and $X$) and their results combined to yield the total QBER under the
attack. The security margin is defined as the difference between the
Eve-induced QBER and the no-Eve channel QBER.

\subsubsection{Scintillation Fading Analysis}

For each (scenario, turbulence regime) pair, $N = 2000$ independent
samples of the instantaneous transmittance are drawn from a log-normal
distribution:
\begin{equation}
    \eta_i = \eta_\mathrm{mean} \cdot \exp\!\left(\mu_\mathrm{ln}
    + \sigma_\mathrm{ln}\,Z_i\right), \quad Z_i \sim \mathcal{N}(0,1),
    \label{eq:lognormal_sample}
\end{equation}
where $\sigma_\mathrm{ln}^2 = \ln(1 + \mathrm{SI})$ and
$\mu_\mathrm{ln} = -\sigma_\mathrm{ln}^2/2$ \cite{andrews2005laser}.
Samples are clipped to $[0, 1]$. For each sample $\eta_i$, the QBER and
secure key rate are computed, yielding full probability distributions
rather than single-point estimates. The fraction of time the QBER exceeds
the \SI{11}{\percent} security threshold constitutes the \emph{insecure
fraction}—the proportion of time during which no secret key can be
extracted. A distance sweep from 50~m to 5~km is performed for the clear
summer baseline using 500 Monte Carlo samples per distance point, for each
of three turbulence regimes (SI = 0.1, 0.5, 1.5).

% ============================================================
\section{Results}
\label{sec:results}

\subsection{Atmospheric Transmission Modelling}

The libRadtran simulations yield the extinction coefficients summarised in
Table~\ref{tab:kext}.

\begin{table}[H]
\centering
\caption{Extinction coefficients at \SI{1550}{nm} for three Scottish
atmospheric scenarios, derived from libRadtran MYSTIC Monte Carlo
simulations. Transmittances are evaluated at \SI{500}{m} and \SI{1}{km}
deployment distances.}
\label{tab:kext}
\begin{tabular}{lccccc}
\toprule
Scenario & $k_\mathrm{ext}$ (km$^{-1}$) & $\eta$ @ 500~m & $\eta$ @ 1~km
& $\eta$ @ 10~km & Dominant loss \\
\midrule
Baseline (Clear Summer) & 0.0360 & 0.982 & 0.964 & 0.698 & Aerosol (Mie) \\
Scottish (Clear Winter) & 0.0398 & 0.980 & 0.961 & 0.670 & Aerosol (Mie) \\
Scottish (Mist/Drizzle) & 0.2368 & 0.888 & 0.789 & 0.094 & Strong Mie \\
\bottomrule
\end{tabular}
\end{table}

Figure~\ref{fig:transmission} shows the full transmittance curves for each
scenario over 0--20~km. The clear summer and winter profiles are closely
matched at short range, diverging only beyond $\sim$\SI{5}{km}. The mist
scenario departs dramatically from the clear-air cases: at \SI{10}{km} the
mist transmittance is $\sim$9\%, compared with $\sim$70\% in clear air.
This is consistent with the dominance of Mie scattering under reduced
visibility conditions; at \SI{1550}{nm}, water droplets with radii
$r \sim \lambda$ are highly efficient scattering centres
\cite{chan2006free}.

\begin{figure}[H]
    \centering
    \includegraphics[width=0.85\linewidth]{Scottish_Transmission_Comparison.png}
    \caption{Atmospheric transmittance at \SI{1550}{nm} as a function of
    horizontal path length for three Scottish weather scenarios. Curves are
    derived from libRadtran MYSTIC Monte Carlo simulations using equation
    (\ref{eq:beerlam}). The vertical dashed line indicates the \SI{500}{m}
    KORUZA deployment distance. Clear summer ($k_\mathrm{ext} = 0.036$
    km$^{-1}$, blue) and clear winter ($k_\mathrm{ext} = 0.040$
    km$^{-1}$, green) are closely matched; mist ($k_\mathrm{ext} = 0.237$
    km$^{-1}$, red) shows severe attenuation beyond $\sim$2~km.}
    \label{fig:transmission}
\end{figure}

\subsection{BB84 QKD Channel Feasibility}

Figure~\ref{fig:bb84} presents the key BB84 channel metrics for each
atmospheric scenario across 50~m to 5~km. At the \SI{500}{m} deployment
distance, the QBER remains at 1.34--1.38\% across all scenarios
(Table~\ref{tab:qkd_summary}), well below the \SI{11}{\percent}
Shor--Preskill threshold. Even in mist, the transmittance remains
sufficiently high at short range that the dark-count contribution to QBER
is negligible; the dominant QBER source is optical misalignment at
$e_\mathrm{misalign} = 1\%$.

\begin{figure}[H]
    \centering
    \includegraphics[width=\linewidth]{BB84_QKD_Scottish_Scenarios.png}
    \caption{BB84 QKD channel performance versus link distance for three
    Scottish atmospheric scenarios. \textbf{(a)} Atmospheric transmittance
    $\eta_\mathrm{atm}$. \textbf{(b)} Quantum bit error rate (QBER);
    solid lines are the analytical model (equation~\ref{eq:qber_total}),
    circles are Qiskit amplitude-damping circuit simulations. The horizontal
    dotted line marks the \SI{11}{\percent} Shor--Preskill security
    threshold. \textbf{(c)} Sifted key rate. \textbf{(d)} Secure key rate
    computed from equation~(\ref{eq:skr}). The vertical dashed line
    indicates the \SI{500}{m} deployment distance.}
    \label{fig:bb84}
\end{figure}

\begin{table}[H]
\centering
\caption{BB84 QKD performance metrics at the \SI{500}{m} KORUZA deployment
distance for each atmospheric scenario. System parameters: $\mu = 0.1$,
$\eta_\mathrm{det} = 0.15$, $f_\mathrm{rep} = \SI{1}{MHz}$,
$d_\mathrm{rate} = 100$~s$^{-1}$.}
\label{tab:qkd_summary}
\begin{tabular}{lcccc}
\toprule
Scenario & $\eta_\mathrm{atm}$ & QBER (\%) & Sifted rate (bps) & Secure rate (bps) \\
\midrule
Baseline (Clear Summer) & 0.9822 & 1.34 & 7366 & 5856 \\
Scottish (Clear Winter) & 0.9803 & 1.34 & 7352 & 5844 \\
Scottish (Mist/Drizzle) & 0.8883 & 1.38 & 6663 & 5267 \\
\bottomrule
\end{tabular}
\end{table}

The critical transmittance threshold below which no secret key can be
extracted (where $e = 11\%$) corresponds to $\eta_\mathrm{atm} \approx 0.033$,
as shown in Figure~\ref{fig:qber_vs_eta}. In clear air, this threshold is
reached at $\sim$\SI{10}{km}; in mist, at $\sim$\SI{1.5}{km}. The
\SI{500}{m} KORUZA operating range is therefore comfortably within the
secure regime for all three scenarios.

\begin{figure}[H]
    \centering
    \includegraphics[width=0.85\linewidth]{BB84_QBER_vs_Transmittance.png}
    \caption{BB84 QKD performance as a function of channel transmittance
    $\eta$. \textbf{(a)} QBER versus $\eta$; the red dashed line is the
    \SI{11}{\percent} Shor--Preskill security limit and the shaded region
    denotes the insecure regime. Coloured markers indicate the operating
    point of each Scottish scenario at \SI{500}{m}. \textbf{(b)} Secure
    key rate versus $\eta$; the vertical dashed line indicates the critical
    transmittance threshold $\eta \approx 0.033$ below which key
    generation ceases.}
    \label{fig:qber_vs_eta}
\end{figure}

\subsection{Eavesdropper Detection}

Figure~\ref{fig:eve} shows the effect of a full intercept-resend attack on
the QBER and secure key rate. Across all atmospheric scenarios, Eve's
attack raises the QBER from $\sim$1.4\% (no attack) to $\sim$40\% at
\SI{500}{m}. This represents a QBER elevation of $\sim$38.6 percentage
points, far exceeding the \SI{11}{\percent} security threshold and
constituting unambiguous detection. The Qiskit circuit simulations confirm
this behaviour, producing QBER values consistent with the theoretical
expectation that an intercept-resend attack introduces a 25\% QBER on the
error-free sifted bits.

The security margin (the difference between Eve's QBER and the channel
noise QBER) is largest at short range and high transmittance. As the link
distance increases and atmospheric attenuation raises the baseline QBER,
this margin gradually narrows. However, at the \SI{500}{m} deployment
range, the margin of $\sim$38\% is such that even a partial attack (Eve
intercepting a fraction of pulses) would be detectable provided she
intercepts more than $\sim$3\% of transmitted qubits.

\begin{figure}[H]
    \centering
    \includegraphics[width=\linewidth]{Eve_Attack_BB84_FSO.png}
    \caption{Effect of a full intercept-resend eavesdropper attack on BB84
    QKD performance. \textbf{(a)} QBER with (bold) and without (faded) Eve;
    all scenario pairs lie well above the \SI{11}{\percent} security limit
    under attack. \textbf{(b)} QBER with Eve from Qiskit circuit simulation.
    \textbf{(c), (d)} Secure key rate without and with Eve, respectively;
    the attack reduces the key rate to zero in all scenarios by driving QBER
    above threshold.}
    \label{fig:eve}
\end{figure}

\subsection{Scintillation Fading Analysis}

Figure~\ref{fig:scint_dist} shows the probability distributions of QBER
under log-normal scintillation fading at the \SI{500}{m} deployment
distance. For the clear summer baseline, weak turbulence (SI = 0.1)
produces a narrow QBER distribution tightly centred near 1.4\%, with the
\SI{95}{\text{th}} percentile at 1.58\% and an insecure fraction of
\SI{0.0}{\percent}. Moderate turbulence (SI = 0.5) broadens the
distribution, raising the 95th percentile to 2.19\%, still well within the
secure regime. Strong turbulence (SI = 1.5) produces a substantially wider
distribution; the 95th percentile QBER is 3.84\% and the insecure fraction
is 0.1\%. These results are summarised in Table~\ref{tab:scint_summary}.

\begin{figure}[H]
    \centering
    \includegraphics[width=\linewidth]{Scintillation_QBER_Distributions.png}
    \caption{Probability distributions of QBER under log-normal scintillation
    fading at the \SI{500}{m} deployment distance. Rows correspond to
    atmospheric scenarios (clear summer, mist); columns to turbulence
    regimes (weak SI=0.1, moderate SI=0.5, strong SI=1.5). Histograms
    represent $N=2000$ Monte Carlo samples; solid curves are log-normal
    fits. The vertical red dashed line is the \SI{11}{\percent} Shor--Preskill
    threshold. Navy circles on the baseline indicate QBER values from
    Qiskit amplitude-damping circuit simulations at representative
    transmittance levels.}
    \label{fig:scint_dist}
\end{figure}

\begin{table}[H]
\centering
\caption{Scintillation fading statistics at \SI{500}{m} deployment
distance. Insecure fraction is the probability that instantaneous QBER
exceeds \SI{11}{\percent}. Log-normal fading model, $N=2000$ Monte Carlo
samples, seed fixed for reproducibility.}
\label{tab:scint_summary}
\begin{tabular}{llcccc}
\toprule
Scenario & Turbulence & SI & Mean QBER (\%) & 95th pct (\%) & Insecure (\%) \\
\midrule
Clear Summer & Weak     & 0.1 & 1.40 & 1.58 & 0.00 \\
             & Moderate & 0.5 & 1.55 & 2.19 & 0.00 \\
             & Strong   & 1.5 & 1.93 & 3.84 & 0.10 \\
\midrule
Mist/Drizzle & Weak     & 0.1 & 1.43 & 1.67 & 0.00 \\
             & Moderate & 0.5 & 1.59 & 2.32 & 0.00 \\
             & Strong   & 1.5 & 2.00 & 3.81 & 0.40 \\
\bottomrule
\end{tabular}
\end{table}

Figure~\ref{fig:scint_sweep} shows how mean QBER, insecure fraction, and
mean secure key rate evolve with distance for each turbulence regime. Up to
$\sim$\SI{1}{km}, all three turbulence regimes keep the mean QBER well below
the security threshold. Beyond $\sim$\SI{2}{km}, strong turbulence drives
the 95th-percentile QBER above threshold and the insecure fraction rises
rapidly. At \SI{5}{km}, strong turbulence produces a mean QBER approaching
4\% and an insecure fraction exceeding 10\%. The mean secure key rate
remains positive under all conditions within the KORUZA deployment range.

\begin{figure}[H]
    \centering
    \includegraphics[width=\linewidth]{Scintillation_Distance_Sweep.png}
    \caption{FSO QKD performance versus link distance under log-normal
    scintillation fading. Clear summer baseline ($k_\mathrm{ext} = 0.036$
    km$^{-1}$). \textbf{(a)} Mean QBER (solid lines) with shaded band
    extending to the 95th percentile; the dotted horizontal line marks the
    \SI{11}{\percent} security threshold. \textbf{(b)} Fraction of time
    QBER exceeds threshold (link insecure). \textbf{(c)} Mean secure key
    rate. The vertical dashed line in each panel marks the \SI{500}{m}
    deployment distance.}
    \label{fig:scint_sweep}
\end{figure}

% ============================================================
\section{Discussion}
\label{sec:discussion}

\subsection{Atmospheric Channel at 1550~nm}

The libRadtran results confirm that \SI{1550}{nm} is well-suited to
short-range terrestrial FSO in Scottish conditions. The clear winter
scenario has a marginally higher extinction coefficient than clear summer
(0.040 vs.\ 0.036~km$^{-1}$), consistent with the expectation that drier
winter air slightly reduces water vapour absorption, but a reduced
visibility (20 vs.\ 23~km) raises aerosol scattering sufficiently to offset
this benefit. This finding is consistent with Grabner and Kvicera
\cite{grabner2014} who found that aerosol loading, rather than molecular
absorption, governs \SI{1550}{nm} attenuation on typical European
horizontal paths.

The mist scenario is far more significant for link reliability: a
$6\times$ increase in $k_\mathrm{ext}$ relative to clear air results in
transmittances below 10\% beyond \SI{10}{km}. Mie scattering is
particularly effective at \SI{1550}{nm} because typical fog and mist
droplets have radii of $1$--$10$~\textmu m, comparable to the wavelength.
This confirms that high-reliability FSO links in Scotland would require
either shorter hop distances, relay-assisted multi-hop architectures
\cite{nor2017}, or backup RF links during adverse weather
events, consistent with the conclusions of \cite{khalighi2014survey} for
European climates.

\subsection{QKD Feasibility at 500~m}

All three atmospheric scenarios are unambiguously within the secure QKD
operating regime at the \SI{500}{m} KORUZA deployment range. The dominant
QBER contribution is optical misalignment ($e_\mathrm{misalign} = 1\%$);
dark counts and atmospheric loss make negligible contributions at this
range. This highlights the importance of the PAT system in a practical
deployment: even a misalignment of a few milliradians would raise
$e_\mathrm{misalign}$ substantially, consuming the available secure key
margin. The KORUZA stepper-motor PAT system provides coarse alignment;
fine-pointing feedback from received optical power monitoring would be
required for a long-term deployment. Farid and Hranilovic
\cite{farid2007} showed that optimising the transmitted beam width with
respect to detector aperture and pointing-error variance can increase the
achievable rate by up to 80\% at a given outage probability under fog
conditions, providing theoretical motivation for this fine-alignment
requirement.

The secure key rates of 5.3--5.9~kbps at \SI{500}{m} are competitive with
the state-of-the-art metropolitan FSO QKD demonstrations. Kržič et al.\
\cite{krzic2023metropolitan} achieved 5.7~kbps at night over a \SI{1.7}{km}
entanglement-based link using significantly more sophisticated hardware.
The KORUZA-based system operating at shorter range with a simpler
prepare-and-measure scheme achieves comparable rates, illustrating the
advantage of reduced path length.

\subsection{Security Against Eavesdropping}

The eavesdropper simulation confirms that an intercept-resend attack is
highly detectable at \SI{500}{m}: the QBER elevation of $\sim$38 percentage
points far exceeds the 11\% threshold. This strong detectability arises
primarily from the low channel QBER baseline (1.3--1.4\%). Eve cannot
exploit atmospheric noise to conceal her attack because the noise level is
simply too low at this range.

It is worth noting that this analysis assumes an ideal intercept-resend
strategy. Coherent attacks—where Eve stores quantum states in a quantum
memory and performs joint measurements—are theoretically more efficient
but substantially harder to implement. For the short-range metropolitan
scenario of this project, the intercept-resend bound is the operationally
relevant threat model \cite{pirandola2020advances,xu2020}.

\subsection{Impact of Scintillation on QKD Security}

The scintillation fading analysis reveals a physically important nuance
absent from the fixed-transmittance model: even when the mean QBER is
comfortably within the secure regime, turbulence-induced fading creates
brief windows during which the instantaneous QBER temporarily exceeds the
security threshold. At \SI{500}{m} under strong turbulence (SI = 1.5),
this occurs 0.1\% of the time in clear air and 0.4\% in mist. While small,
these intervals are operationally meaningful: during them, any key
generation attempt would produce insecure bits that must be discarded
in post-processing. Adaptive protocols that pause key generation when a
received-power monitor indicates deep fades can mitigate this at the cost
of some key rate reduction.

The scintillation index values used here (0.1--1.5) span the range
documented for horizontal terrestrial FSO paths of comparable length and
wavelength \cite{andrews2005laser}. However, actual SI values for the
specific deployment site should be measured from the temporal statistics
of the received optical power time series during experimental testing—a
direct output of the KORUZA monitoring system once deployed.

\subsection{Proposed Hardware Adaptations for QKD}

The present KORUZA implementation establishes the classical FSO channel.
To implement BB84 QKD, the following hardware modifications would be
required. At the transmitter, the SFP-based coherent laser source would be
replaced by an attenuated pulsed laser at $\mu \sim 0.1$ photons per pulse,
with a polarisation encoder (half-wave plate and polarising beamsplitter)
to prepare the four BB84 states. At the receiver, the SFP detector and
media converter would be replaced by a polarisation analysis module (50:50
beamsplitter and two polarising beamsplitters feeding four single-photon
avalanche diodes). The Raspberry Pi control and PAT hardware would remain
unchanged, since pointing and alignment is the primary engineering
challenge—as this work has characterised. This architecture is closely
analogous to that deployed by the HWU quantum communications group
\cite{krzic2023metropolitan, simmons2024polarisation, ocampos2019}.

% ============================================================
\section{Retroreflector Stability Characterisation}
\label{sec:retro}
\textit {To be done}
\subsection{Experimental Setup}

\textit{[Placeholder}

The constructed KORUZA unit was tested against a corner-cube retroreflector
placed at a distance of \todo{distance (m)} to characterise the pointing
stability and received power statistics of the PAT system under static
conditions. The retroreflector was positioned at \todo{location description}.
Received optical power was logged via \todo{monitoring method} over a
duration of \todo{measurement duration}.

\subsection{Received Power Statistics}

\begin{table}[H]
\centering
\caption{Received power statistics from retroreflector stability test.
All values to be filled from experimental data.}
\label{tab:retro_power}
\begin{tabular}{lc}
\toprule
Metric & Value \\
\midrule
Mean received power (dBm)        & \textit{[TBD]} \\
Standard deviation (dB)          & \textit{[TBD]} \\
Minimum received power (dBm)     & \textit{[TBD]} \\
Maximum received power (dBm)     & \textit{[TBD]} \\
Measurement duration             & \textit{[TBD]} \\
Sampling rate                    & \textit{[TBD]} \\
Estimated scintillation index    & \textit{[TBD]} \\
\bottomrule
\end{tabular}
\end{table}

\subsection{Pointing Stability}

\begin{figure}[H]
    \centering
    \fbox{\parbox{0.85\linewidth}{\centering\vspace{2cm}
    \textit{[Figure placeholder}
    \vspace{2cm}}}
    \caption{Received optical power time series during retroreflector
    stability test at \todo{distance}~m. The dashed line indicates the mean
    received power; shading denotes $\pm 1\sigma$. }
    \label{fig:retro_power}
\end{figure}

The empirical scintillation index, computed from the time series as
$\mathrm{SI} = (\langle P^2 \rangle - \langle P \rangle^2) / \langle P \rangle^2$,
will be compared against the log-normal model values of 0.1--1.5 used in
Section~\ref{sec:results}. Any systematic drift in received power will
indicate slow beam wander, while high-frequency fluctuations correspond
to scintillation.

\subsection{Discussion of Experimental Results}

\textit{[To be written after data collection. Key points to address:
agreement or discrepancy with libRadtran transmittance predictions;
observed scintillation index vs.\ simulation inputs; PAT system performance
and any pointing corrections applied during the measurement; implications
for the QKD feasibility conclusions of Section~\ref{sec:discussion}.]}

% ============================================================
\section{Conclusions}
\label{sec:conclusions}

This project has constructed and characterised a low-cost, open-source FSO
terminal based on the IRNAS KORUZA platform, with the original custom PCB
replaced by off-the-shelf components. Three complementary computational
analyses have been performed to characterise both the classical atmospheric
channel and its quantum communication implications.

Atmospheric modelling using libRadtran MYSTIC Monte Carlo simulations
confirms that \SI{1550}{nm} is well-suited to short-range FSO in Scottish
conditions. The clear-air extinction coefficient of 0.036--0.040~km$^{-1}$
yields transmittances above 98\% at \SI{500}{m}; mist conditions raise
$k_\mathrm{ext}$ by a factor of six but leave the \SI{500}{m} link with
88\% transmittance.

Qiskit BB84 simulations show that all three atmospheric scenarios support
secure QKD at \SI{500}{m} with QBER values of 1.34--1.38\% and secure key
rates of 5.3--5.9~kbps. An intercept-resend eavesdropper attack is
unambiguously detectable at this range, elevating QBER to $\sim$40\%. Log-normal
scintillation analysis further reveals that strong turbulence (SI = 1.5)
causes the link to enter the insecure regime for at most 0.4\% of the time
at \SI{500}{m}, a figure that rises rapidly beyond \SI{2}{km}.

The results collectively confirm that the KORUZA FSO terminal, operating
at \SI{500}{m}, provides a viable classical channel for metropolitan QKD.
The key outstanding engineering step is the replacement of the SFP data
link with a single-photon source and polarisation analysis module, while
retaining the Raspberry Pi PAT system that this work has validated.

\subsection{Further Work}

Immediate next steps include experimental deployment of the constructed
unit to measure actual BER and received power time series, from which
empirical scintillation indices and transmittance values can be extracted
and compared with the libRadtran predictions. Longer-term work could
investigate adaptive-optics tip/tilt correction (using a fast steering
mirror driven by the Raspberry Pi) to suppress beam wander, and the
integration of attenuated-pulse QKD hardware as outlined in
Section~\ref{sec:discussion}. The HWU Qrackling simulation framework
\cite{qrackling} provides a natural platform for extending the quantum
channel analysis to entanglement-based BBM92 protocols, which may offer
security advantages in the metropolitan FSO context.

% ============================================================
\section*{Acknowledgements}

The author thanks Alessandro Fedrizzi for guidance throughout this
project, and Faris Redza (PhD candidate, Heriot-Watt University) for
technical support and valuable discussions during the hardware construction
and alignment phase. The libRadtran software package is maintained by the
Ludwig Maximilian University of Munich and the Norwegian Institute for Air
Research; its open availability was essential to this work. The Qiskit
open-source quantum computing framework is maintained by IBM.

% ============================================================
\begin{thebibliography}{99}

\bibitem{khalighi2014survey}
Khalighi, M.~A. \& Uysal, M.
Survey on free space optical communication: a communication theory
perspective.
\textit{IEEE Commun.\ Surv.\ Tutor.}\ \textbf{16}, 2231--2258 (2014).

\bibitem{barrios2012wireless}
Barrios, R. \& Dios, F.
Wireless optical communications through the turbulent atmosphere: a review.
In \textit{Optical Communications Systems} (ed.\ Das, N.) (InTech, 2012).

\bibitem{chan2006free}
Chan, V.~W.~S.
Free-space optical communications.
\textit{J.\ Lightwave Technol.}\ \textbf{24}, 4750--4762 (2006).

\bibitem{fante1975electromagnetic}
Fante, R.~L.
Electromagnetic beam propagation in turbulent media.
\textit{Proc.\ IEEE}\ \textbf{63}, 1669--1692 (1975).

\bibitem{tatarskii1961}
Tatarskii, V.~I.
\textit{Wave Propagation in a Turbulent Medium} (McGraw-Hill, 1961).

\bibitem{andrews2005laser}
Andrews, L.~C. \& Phillips, R.~L.
\textit{Laser Beam Propagation through Random Media}, 2nd edn (SPIE Press,
2005).

\bibitem{emde2016libradtran}
Emde, C. et al.
The libRadtran software package for radiative transfer calculations
(version 2.0.1).
\textit{Geosci.\ Model Dev.}\ \textbf{9}, 1647--1672 (2016).

\bibitem{bennett1984}
Bennett, C.~H. \& Brassard, G.
Quantum cryptography: public key distribution and coin tossing.
In \textit{Proc.\ IEEE Int.\ Conf.\ Computers, Systems and Signal
Processing}, 175--179 (IEEE, 1984).

\bibitem{shor2000simple}
Shor, P.~W. \& Preskill, J.
Simple proof of security of the BB84 quantum key distribution protocol.
\textit{Phys.\ Rev.\ Lett.}\ \textbf{85}, 441--444 (2000).

\bibitem{gobby2004quantum}
Gobby, C., Yuan, Z.~L. \& Shields, A.~J.
Quantum key distribution over 122~km of standard telecom fiber.
\textit{Appl.\ Phys.\ Lett.}\ \textbf{84}, 3762--3764 (2004).

\bibitem{nielsen2010quantum}
Nielsen, M.~A. \& Chuang, I.~L.
\textit{Quantum Computation and Quantum Information} (Cambridge University
Press, 2010).

\bibitem{krzic2023metropolitan}
Kržič, A. et al.
Towards metropolitan free-space quantum networks.
\textit{npj Quantum Inf.}\ \textbf{9}, 95 (2023).

\bibitem{erven2008entangled}
Erven, C., Couteau, C., Laflamme, R. \& Weihs, G.
Entangled quantum key distribution over two free-space optical links.
\textit{Opt.\ Express}\ \textbf{16}, 16840--16853 (2008).

\bibitem{koruza}
IRNAS.
KORUZA: open-source wireless optical system.
\textit{https://www.irnas.eu/koruza/} (2016).

\bibitem{anthonisamy2016performance}
Anthonisamy, A.~B.~R., Durairaj, P. \& Paul, L.~J.
Performance analysis of free space optical communication in open-atmospheric
turbulence conditions with beam wandering compensation control.
\textit{IET Commun.}\ \textbf{10}, 1096--1103 (2016).

\bibitem{simmons2024polarisation}
Simmons, C., Barrow, P. \& Donaldson, R.
Dawn and dusk satellite quantum key distribution using time and
phase-based encoding and polarization filtering.
Preprint at \textit{arXiv:2401.03129} (2024).

\bibitem{grabner2014}
Grabner, M. \& Kvicera, V.
The fog attenuation dependence on visibility and liquid water content
at \SI{850}{nm} and \SI{1550}{nm} wavelength.
\textit{Radioengineering}\ \textbf{23}, 733--740 (2014).

\bibitem{pirandola2020advances}
Pirandola, S. et al.
Advances in quantum cryptography.
\textit{Adv.\ Opt.\ Photon.}\ \textbf{12}, 1012--1236 (2020).

\bibitem{qiskit}
Qiskit contributors.
Qiskit: an open-source framework for quantum computing.
\textit{https://doi.org/10.5281/zenodo.2573505} (2023).

\bibitem{qrackling}
Simmons, C., Barrow, P. \& Donaldson, R.
Qrackling: satellite QKD simulation framework.
\textit{https://github.com/Free-Space-QKD-Lab-HWU/Qrackling} (2024).

\bibitem{al-habash2001}
Al-Habash, M.~A., Andrews, L.~C. \& Phillips, R.~L.
Mathematical model for the irradiance probability density function of a
laser beam propagating through turbulent media.
\textit{Opt.\ Eng.}\ \textbf{40}, 1554--1562 (2001).

\bibitem{farid2007}
Farid, A.~A. \& Hranilovic, S.
Outage capacity optimisation for free-space optical links with pointing
errors.
\textit{J.\ Lightwave Technol.}\ \textbf{25}, 1702--1710 (2007).

\bibitem{motlagh2008}
Chaman Motlagh, A., Ahmadi, V., Ghassemlooy, Z. \& Abedi, K.
The effect of atmospheric turbulence on the performance of the free space
optical communications.
In \textit{Proc.\ 6th Symp.\ Communication Systems, Networks and Digital
Signal Processing (CSNDSP)}, 540--543 (IEEE, 2008).

\bibitem{lo2005}
Lo, H.-K., Ma, X. \& Chen, K.
Decoy state quantum key distribution.
\textit{Phys.\ Rev.\ Lett.}\ \textbf{94}, 230504 (2005).

\bibitem{xu2020}
Xu, F., Ma, X., Zhang, Q., Lo, H.-K. \& Pan, J.-W.
Secure quantum key distribution with realistic devices.
\textit{Rev.\ Mod.\ Phys.}\ \textbf{92}, 025002 (2020).

\bibitem{liao2017}
Liao, S.-K. et al.
Satellite-to-ground quantum key distribution.
\textit{Nature}\ \textbf{549}, 43--47 (2017).

\bibitem{bedington2017}
Bedington, R., Arrazola, J.~M. \& Ling, A.
Progress in satellite quantum key distribution.
\textit{npj Quantum Inf.}\ \textbf{3}, 30 (2017).

\bibitem{nor2017}
Mohd Nor, N.~A. et al.
Experimental investigation of all-optical relay-assisted 10~Gb/s FSO link
over the atmospheric turbulence channel.
\textit{J.\ Lightwave Technol.}\ \textbf{35}, 45--53 (2017).

\bibitem{ocampos2019}
Ocampos-Guill\'{e}n, A., Denisenko, N. \& Fern\'{a}ndez-M\'{a}rmol, V.
Optimising the interconnection of free-space to fibre quantum networks.
\textit{EPJ Web Conf.}\ \textbf{198}, 00007 (2019).

\bibitem{mohamedfso2023}
Mohamed, P.~H., El-Shimy, M.~A., Shalaby, H.~M.~H. \& Kheirallah, H.~N.
FSO channel modelling and performance evaluation over dust combined with
Gamma-Gamma atmospheric turbulence.
In \textit{Proc.\ 40th National Radio Science Conf.\ (NRSC)}, 121--130
(IEEE, 2023).

\end{thebibliography}

% ============================================================
\appendix

\section{Risk Assessment}
\label{app:risk}

The risk assessment submitted at the start of Semester 1 is reproduced
below. Four hazards were identified: electrical equipment (risk score 3),
display screen equipment (risk score 2), moving parts from the motor
alignment sequence (risk score 2), and Class~1M laser radiation (residual
risk score 8 after controls). Control measures include laser safety
training, warning signs, power interlock, and prohibition on direct beam
viewing. Full details are given in the project registration documentation.

\section{Simulation Code}
\label{app:code}

All simulation scripts are available in the project GitHub repositories:
\begin{itemize}
    \item Atmospheric modelling: \texttt{github.com/Toddb2/Libradtran-HW}
    (scripts: \texttt{plot\_data\_scotland.py}, \texttt{plot\_data3.py})
    \item QKD channel simulations: \texttt{qkd\_bb84\_simulation.py},
    \texttt{eve\_attack\_simulation.py},
    \texttt{scintillation\_simulation.py}
\end{itemize}
Python dependencies: \texttt{qiskit==2.3.0}, \texttt{qiskit-aer},
\texttt{numpy}, \texttt{matplotlib}, \texttt{scipy}.

\end{document}